%========================================================
%Proceso
%========================================================

%========================================================
% Descripción general del proceso
%-----------------------------------------------
\begin{Proceso}{P5.1}{Gestion de credenciales} {
  
  %-------------------------------------------
  %Resumen

  Proceso que se encarga de la gestión de credenciales, en donde se genera las credenciales para lectores \cdtRef{Actor:Estudiante}{Estudiante} y \cdtRef{Actor:Profesor}{Profesor}. Además de renovar la fecha de vigencia de las credenciales para poder seguir haciendo uso de esta.

  %-------------------------------------------
  %Diagrama del proceso

  \noindent La Figura \cdtRefImg{P5.1}{Gestion de Credenciales} muestra las actividades que se realizan para llevar a cabo el proceso descrito anteriormente.

  \Pfig[0.95]{./procesos/C5_Credenciales/Images/PA5_1-GestionDeCredenciales.png}{P5.1}{Gestion de Credenciales}

} {P5.1:Gestion}

  %-------------------------------------------
  %Elementos del proceso

  \UCitem{Actores} { %Actores
    \cdtRef{Actor:Lector}{Lector}, \cdtRef{Actor:Gestion de Usuarios}{Gestion de Usuarios} y \cdtRef{Actor:Gestion de Credenciales}{Gestion de Credenciales}.
  }

  \UCitem{Objetivo} { %Objetivo
    Gestión de credenciales
  }

  \UCitem{Insumos de entrada} { %Insumos de entrada
      ID de Lector
  }
  
  \UCitem{Proveedores} { %Proveedores
    Gestión de usuarios y usuario.
  }

  \UCitem{Productos de salida} { %Productos de salida
    Formado de Credencial
  }

  \UCitem{Cliente} { %Cliente
    Gestión de usuarios
  }

  \UCitem{Mecanismo de medición} { %Mecanismo de medición
    No se identificaron mecanismos de medición
  }
  \UCitem{Interrelación con otros procesos} { %Interrelación con otros procesos
    Gestión de usuarios
  }
  \UCitem{Tipo de solicitud al que aplica el proceso} { %Tipo de solicitud al que aplica el proceso
    Usuarios que necesiten la interacción de credenciales.
  }


\end{Proceso}

%========================================================
%Descripción de tareas
%-----------------------------------------------
\begin{PDescripcion}

  %Actor: Lector
  \Ppaso Lector

    \begin{enumerate}

      %Tarea a
      \Ppaso[\itarea] \cdtLabelTask{T1-P5.1:Lector.}{Solicita registro.} El usuario solicita al bibliotecario registrarse en el sistema bibliotecario de la Escuela Superior de Cómputo y proporciona a \cdtRef{Actor:Gestion de Usuarios}{Gestion de Usuarios} sus datos personales.

      %Tarea b
      \Ppaso[\itarea] \cdtLabelTask{T2-P5.1:Lector.}{Solicita reposición.} El usuario solicita la reposición de su credencial a \cdtRef{Actor:Gestion de Usuarios}{Gestion de Usuarios}.

    \end{enumerate}
    
    %Actor: Gestión de usuarios
  \Ppaso Gestion De Usuarios

    \begin{enumerate}

      %Tarea a
      \Ppaso[\itarea] \cdtLabelTask{T1-P5.1:GestionDeUsuarios.}{Registrar usuario.} Gestión de usuarios se encarga de registrar un nuevo usuario.

      %Tarea b
      \Ppaso[\itarea] \cdtLabelTask{T2-P5.1:GestionDeUsuarios.}{Enviar ID.} Espera ser invocado por la petición del usuario para renovar credencial o la petición de registrar usuario para que así pueda enviar el Identificador del usuario que invoco a la tarea predecesora de esta.

    \end{enumerate}
    
    %Actor: Gestión de Credenciales
  \Ppaso Gestion de Credenciales

    \begin{enumerate}

      %Tarea a
      \Ppaso[\itarea] \cdtLabelTask{T1-P5.1:GestionDeCredenciales.}{Generar credencial.} \cdtRef{Actor:Gestion de Credenciales}{Gestion de Credenciales} obtiene todos los datos del lector al cual se le generara una credencial por medio del identificador de lector que recibe y posteriormente genera la credencial que se le asignara al lector.

      %Tarea b
      \Ppaso[\itarea] \cdtLabelTask{T2-P5.1:GestionDeCredenciales.}{Renovación de credencial.} Se renovara la vigencia de la credencial para que pueda seguir operando.

    \end{enumerate}

\end{PDescripcion}
