%========================================================
%Proceso
%========================================================

%========================================================
% Descripción general del proceso
%-----------------------------------------------
\begin{Proceso}{P5.3}{Solicitud de renovación de credencial} {
  
  %-------------------------------------------
  %Resumen

  Proceso que realiza el \cdtRef{Actor:Lector}{Lector} para generarle una credencial, y así poder hacer uso de los servicios bibliotecarios.
  Si el periodo de renovación se encuentra vigente, el sistema permite al \cdtRef{Actor:Aspirante}{Aspirante} ingresar su ID de lector [la boleta del alumno], de lo contrario le notifica que se encuentra fuera de dicho periodo y por lo tanto no se pueden solicitarla al sistema. Cuando se ingresa el ID de lector, el sistema proporciona los datos de Usuario.
  Una vez que la información es enviada, el Sistema debe asegurarse de que el \cdtRef{Actor:Estudiante}{Estudiante} realice el pago, en caso de no hacerlo el proceso termina; por otro lado, si procede con el pago de la credencial el interesado deberá esperar a que el sistema le notifique “Credencial lista” momento en que el Lector puede ir a recoger la credencial.
  La Figura 1 Generación de credencial muestra las actividades que se realizan para llevar a cabo el proceso descrito anteriormente.

  %-------------------------------------------
  %Diagrama del proceso

  \noindent La Figura \cdtRefImg{P5.3}{Generacion de Credencial} muestra las actividades que se realizan para llevar a cabo el proceso descrito anteriormente.

  \Pfig[0.95]{./procesos/C5_Credenciales/Images/PA5_3-GeneracionDeCredencial.png}{P5.3}{Generacion de Credencial}

} {P5.3:Cuenta}

  %-------------------------------------------
  %Elementos del proceso

  \UCitem{Actores} { %Actores
    \cdtRef{Actor:Usuario}{Usuario} y \cdtRef{Actor:Bibliotecario}{Bibliotecario}
  }

  \UCitem{Objetivo} { %Objetivo
    Generar la credencial del usuario
  }

  \UCitem{Insumos de entrada} { %Insumos de entrada
    ID de lector
  }
  
  \UCitem{Proveedores} { %Proveedores
    \cdtRef{Actor:Lector}{Lector}
  }

  \UCitem{Productos de salida} { %Productos de salida
    Notificación MSG5.06 Credencial lista
  }

  \UCitem{Cliente} { %Cliente
    \cdtRef{Actor:Lector}{Lector}
  }

  \UCitem{Mecanismo de medición} { %Mecanismo de medición
    No se identificaron mecanismos de medición
  }
  \UCitem{Interrelación con otros procesos} { %Interrelación con otros procesos
    Gestión de usuarios
  }
  \UCitem{Tipo de solicitud al que aplica el proceso} { %Tipo de solicitud al que aplica el proceso
    Usuarios que deseen obtener una credencial de la Biblioteca
  }


\end{Proceso}

%========================================================
%Descripción de tareas
%-----------------------------------------------
\begin{PDescripcion}

  %Actor: Lector
  \Ppaso Lector

    \begin{enumerate}

      %Tarea a
      \Ppaso[\itarea] \cdtLabelTask{T1-P5.3:Lector.}{Proporcionar ID.}  El \cdtRef{Actor:Lector}{Lector} proporciona su ID de lector [la boleta del alumno se tomará como su ID
      de lector].

      %Tarea b
      \Ppaso[\itarea] \cdtLabelTask{T2-P5.3:Lector.}{Realiza pago credencial.} Una vez que el usuario haya recibido el precio de la
      credencial, decidirá pagarla o no, si decide hacerlo, realizará el pago de la misma y posteriormente solicitará al bibliotecario la creación de credencial; si decide no hacerlo, el proceso termina.

      %Tarea c
      \Ppaso[\itarea] \cdtLabelTask{T2-P5.3:Lector.}{Recoge Credencial.} Una vez que el usuario recibe la notificación MSG5.06 “Credencial lista”, éste recoge su credencial y el proceso termina exitosamente.

    \end{enumerate}

  %Actor: Bibliotecario
  \Ppaso Bibliotecario

    \begin{enumerate}

      %Tarea a
      \Ppaso[\itarea] \cdtLabelTask{T1-P5.3:Bibliotecario.}{Creación de credencial.} Una vez que haya recibido el ID de lector del \cdtRef{Actor:Alumno}{Alumno}, el \cdtRef{Actor:Bibliotecario}{Bibliotecario} verifica que se cumplan las siguientes condiciones:

      %Eventos
      \begin{itemize}
        %Evento 1
        \item RN5.02 Pago de Credencial. El bibliotecario pasa a la tarea Obtiene Precio Credencial.
        %Evento 2
        \item RN5.18 Días de Tramites. El bibliotecario pasa a la tarea Obtiene días de trámite.
        %Evento 3
        \item RN5.19 Horario de Atención Trámites. El bibliotecario pasa a la tarea Obtiene horario de atención trámites.
        %Evento 4
        \item RN5.19 RN5.20 Meses de Tramites. El bibliotecario pasa a la tarea Obtiene meses trámites.
      \end{itemize}
      
      Si se cumplen las condiciones anteriores, el bibliotecario crea la credencial y envía el mensaje MSG5.06 Credencial lista. El proceso termina exitosamente.

      %Tarea b
      \Ppaso[\itarea] \cdtLabelTask{T2-P5.3:Bibliotecario.}{Obtiene precio credencial.} El bibliotecario obtiene el precio de la credencial y envía el mensaje MSG5.02 Precio credencial.

       %Tarea c
      \Ppaso[\itarea] \cdtLabelTask{T2-P5.3:Bibliotecario.}{Obtiene días de trámite.} El bibliotecario obtiene los días de trámite y envía el mensaje MSG5.03 Fuera de días trámite.

      %Tarea d
      \Ppaso[\itarea] \cdtLabelTask{T2-P5.3:Bibliotecario.}{Obtiene horario de atención trámites.} El bibliotecario obtiene el horario de trámite y envía el mensaje MSG5.04 Fuera de horario trámite.

      %Tarea e
      \Ppaso[\itarea] \cdtLabelTask{T2-P5.3:Bibliotecario.}{Obtiene meses trámites.} El bibliotecario obtiene los meses de trámites y envía el mensaje MSG5.05 Fuera de meses trámite.
      
    \end{enumerate}

\end{PDescripcion}