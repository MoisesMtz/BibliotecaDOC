
%---------------------------------------------------------
\section{Reglas de Negocio}
%---------------------------------------------------------

\begin{BussinesRule}{R1.1}{Campos no nulos.} 
	\BRitem[Descripción:] Ningún dato en el formulario puede ser nulo.
	\BRitem[Tipo:] Restricción (validación).
	\BRitem[Nivel:] Obligatorio.
\end{BussinesRule}

%---------------------------------------------------------

\begin{BussinesRule}{R1.2}{Formato general para el registro de un empleado.} 

	\BRitem[Nombre:] Formato del nombre.
	\BRitem[Descripción:] El nombre esta compuesto por:
		\begin{itemize} 
			\item Nombre
			\item Primer apellido
			\item Segundo apellido 
		\end{itemize}
		Todo el nombre debe estar compuestos por letras.\\\\
	Ejemplo \\
	Nombre: Luis Ángel, Apellido Paterno: Martínez, Apellido Materno: Gómez.
	\BRitem[Tipo:] Restricción(validación)
	\BRitem[Nivel:] Obligatorio.\\\\

	\BRitem[Telefono:] Formato del telefono.
	\BRitem[Descripción:] El teléfono debe de estar formado solamente con números.\\\\
	Ejemplo \\
	Telefono: 58601859 o 5572753650
	\BRitem[Tipo:] Restricción (validación).
	\BRitem[Nivel:] Obligatorio.\\\\

	\BRitem[No. de empleado IPN:] Formato del numero de empleado del IPN.
	\BRitem[Descripción:] El número del empleado esta compuesto por:
		\begin{itemize} 
			\item Año de ingreso del docente(4 dígitos).
			\item Dos dígitos para saber de donde es egresado(01 egresado del IPN, 10 egresado de otra universidad).
			\item Número de trabajador(4 dígitos).
		\end{itemize}
	Ejemplo:\\
		Año de ingreso: 2014\\
		Egresado de ESCOM: 01\\
		No. trabajador: 3147\\
		Su número de trabajador: 2014013147\\
	\BRitem[Tipo:] Restricción (validación).
	\BRitem[Nivel:] Obligatorio.\\\\

	\BRitem[Rol a desempeñar:] Seleccion de rol.
	\BRitem[Descripción:] Se tiene que seleccionar de una lista el rol a desempeñar:
		\begin{itemize} 
			\item Bibliotecario quien tiene acceso a Prestamos, Lectores, Credencializacion.
			\item Control quien tiene acceso al Inventario de la biblioteca.
		\end{itemize}
	\BRitem[Tipo:] Restricción (validación).
	\BRitem[Nivel:] Obligatorio.\\\\

	\BRitem[Contraseña:] Introduce la contraseña con la que tendra acceso a la gestion del personal.
	\BRitem[Descripción:] La contraseña debe de tener un tamaño mínimo de 8 caracteres y un máximo de 16 caracteres, la cual está compuesta por:
		\begin{itemize}	
			\item Letras mayúsculas.
			\item Letras minúsculas.
			\item Dígitos.
			\item Caracteres no alfanúmericos, es decir, caracteres especiales.
		\end{itemize}
	\BRitem[Tipo:] Restricción (validación).
	\BRitem[Nivel:] Obligatorio.\\\\

	\BRitem[Repetir contraseña:] Introduce la misma contraseña en el paso anterior.
	\BRitem[Descripción:] La contraseña escrita debera ser la misma para confirmar que es la misma.
	\BRitem[Tipo:] Restricción (validación).
	\BRitem[Nivel:] Obligatorio.

\end{BussinesRule}

%---------------------------------------------------------

\begin{BussinesRule}{R1.3}{Duplicidad del personal} 
	\BRitem[Descripción:] No podra haber duplicidad en la informacion del personal, esto incluye mismo numero de empleado y mismo nombre completo. 
	\BRitem[Tipo:] Restricción (validación).
	\BRitem[Nivel:] Obligatorio.
\end{BussinesRule}

%---------------------------------------------------------

\begin{BussinesRule}{R1.4}{Formato de modificacion de datos.} 
	\BRitem[Nombre:] Formato del nombre.
	\BRitem[Descripción:] El nombre esta compuesto por:
		\begin{itemize} 
			\item Nombre
			\item Primer apellido
			\item Segundo apellido 
		\end{itemize}
		Todo el nombre debe estar compuestos por letras.\\\\
	Ejemplo \\
	Nombre: Luis Ángel, Apellido Paterno: Martínez, Apellido Materno: Gómez.
	\BRitem[Tipo:] Restricción(validación)
	\BRitem[Nivel:] Obligatorio.\\\\

	\BRitem[Telefono:] Formato del telefono.
	\BRitem[Descripción:] El teléfono debe de estar formado solamente con números.\\\\
	Ejemplo \\
	Telefono: 58601859 o 5572753650
	\BRitem[Tipo:] Restricción (validación).
	\BRitem[Nivel:] Obligatorio.\\\\

	\BRitem[Rol a desempeñar:] Seleccion de rol.
	\BRitem[Descripción:] Se tiene que seleccionar de una lista el rol a desempeñar:
		\begin{itemize} 
			\item Bibliotecario quien tiene acceso a Prestamos, Lectores, Credencializacion.
			\item Control quien tiene acceso al Inventario de la biblioteca.
		\end{itemize}
	\BRitem[Tipo:] Restricción (validación).
	\BRitem[Nivel:] Obligatorio.\\\\

	\BRitem[Contraseña:] Introduce la contraseña a cambiar.
	\BRitem[Descripción:] La contraseña debe de tener un tamaño mínimo de 8 caracteres y un máximo de 16 caracteres, no podra ser una contraseña usada en un maximo de 1 mes, la cual está compuesta por:
		\begin{itemize}	
			\item Letras mayúsculas.
			\item Letras minúsculas.
			\item Dígitos.
			\item Caracteres no alfanúmericos, es decir, caracteres especiales.
		\end{itemize}
	\BRitem[Tipo:] Restricción (validación).
	\BRitem[Nivel:] Obligatorio.\\\\

	\BRitem[Repetir contraseña:] Introduce la misma contraseña en el paso anterior.
	\BRitem[Descripción:] La contraseña escrita debera ser la misma para confirmar que es la misma.
	\BRitem[Tipo:] Restricción (validación).
	\BRitem[Nivel:] Obligatorio.

\end{BussinesRule}

%---------------------------------------------------------
\begin{BussinesRule}{R1.5}{Contraseña incorrecta.} 
	\BRitem[Descripción:] Solamente la contraseña del Jefe de Biblioteca podra eliminar al personal seleccionado.
	\BRitem[Tipo:] Restricción (validación).
	\BRitem[Nivel:] Obligatorio.
\end{BussinesRule}

%---------------------------------------------------------

\begin{BussinesRule}{RN3.1}{Lector acreedor a Prestamo a Domicilio} 
	\BRitem[Tipo:] Restricción (validación).
	\BRitem[Descripción:]El préstamo a domicilio solo será para lector internos (alumnos, docentes, egresados) registrados en el sistema bibliotecario y para lectores externos que cuenten con su formato firmado y sellado por la biblioteca correspondiente.
\end{BussinesRule}

\begin{BussinesRule}{RN3.2}{Lector con Multas} 
	\BRitem[Tipo:] Restricción (validación).
	\BRitem[Descripción:]El lector (interno o externo) que cuente con multas que estén en calidad de “Sin Pagar”, no podrá sacar material de la biblioteca.
\end{BussinesRule}

\begin{BussinesRule}{RN3.3}{Devoluciones Atrasadas} 
	\BRitem[Tipo:] Restricción (validación).
	\BRitem[Descripción:]El lector (interno o externo) que cuente con material en su posesión y que haya excedido el tiempo límite de entrega no podrá pedir más material en calidad de Préstamo, hasta que entregue dicho material.
\end{BussinesRule}

\begin{BussinesRule}{RN3.4}{Credencial Vigente} 
	\BRitem[Tipo:] Restricción (validación).
	\BRitem[Descripción:]El lector interno que desee realizar algún proceso de la biblioteca (prestamos, devoluciones, prestamos interbibliotecarios) debe de contar con su credencial vigente y actualizada, de lo contrario no podrá hacerlas.
\end{BussinesRule}

\begin{BussinesRule}{RN3.5}{Duración del Préstamo} 
	\BRitem[Tipo:] Restricción (validación).
	\BRitem[Descripción:]La duración del préstamo a domicilio es de 8 días naturales (incluyendo sábados y domingos).
\end{BussinesRule}

\begin{BussinesRule}{RN3.6}{Numero de Prestamos} 
	\BRitem[Tipo:] Restricción (validación).
	\BRitem[Descripción:]El número de material por préstamo está limitado a 3 libros, 2 CDs y un CD de TT, el usuario interno puede seguir pidiendo Material si es que no ha llegado al límite antes dicho y si no tiene multas o devoluciones pendientes.
\end{BussinesRule}

\begin{BussinesRule}{RN3.7}{Prestamo de TT} 
	\BRitem[Tipo:] Restricción (validación).
	\BRitem[Descripción:]El material de TT solo será préstamo por una hora (60 minutos), para la consulta del mismo.
\end{BussinesRule}

\begin{BussinesRule}{RN3.8}{Limite de Prestamo Alcanzado} 
	\BRitem[Tipo:] Restricción (validación).
	\BRitem[Descripción:]Si un usuario quiere pedir algún ejemplar en calidad de préstamo y ya tiene el límite de préstamos en su posesión deberá regresar por lo menos un ejemplar de los que tiene en su poder para realizar dicho préstamo.
\end{BussinesRule}

\begin{BussinesRule}{RN3.9}{Disponibilidad de Libro } 
	\BRitem[Tipo:] Restricción (validación).
	\BRitem[Descripción:]El Libro que solicita el usuario debe estar disponible para su préstamo y dicho ejemplar tener la etiqueta de libro para préstamo.
\end{BussinesRule}

\begin{BussinesRule}{RN3.10}{Numero de Ejemplares} 
	\BRitem[Tipo:] Restricción (validación).
	\BRitem[Descripción:]El lector no puede pedir mas de un ejemplar del mismo tipo en calidad de préstamo.
\end{BussinesRule}

\begin{BussinesRule}{RN3.11}{Observaciones de Material} 
	\BRitem[Tipo:] Restricción (validación).
	\BRitem[Descripción:]En cada proceso de préstamo el bibliotecario revisara y anotara en el sistema las observaciones físicas del material que será prestado para que quede registrado.
\end{BussinesRule}

\begin{BussinesRule}{RN3.12}{Registro de Préstamo} 
	\BRitem[Tipo:] Restricción (validación).
	\BRitem[Descripción:]El préstamo generara un ID el cual se le asociara los datos del usuario, la fecha de realización y el material que se prestó y el estado de estos cambiara a “Prestado”, y se guardara en la BD.
\end{BussinesRule}

\begin{BussinesRule}{RN3.13}{Llenado formulario de Préstamo Interbibliotecario} 
	\BRitem[Tipo:] Restricción (validación).
	\BRitem[Descripción:]Ningún dato en el registro del préstamo puede ser nulo y todos los campos del formulario deben estar llenos.
\end{BussinesRule}

\begin{BussinesRule}{RN3.14}{Formato Interbibliotecario} 
	\BRitem[Tipo:] Restricción (validación).
	\BRitem[Descripción:]Cada Formato de préstamo interbibliotecario debe de contar con un No. de Folio que sea único e irrepetible.
\end{BussinesRule}

\begin{BussinesRule}{RN3.15}{Convenio interbibliotecario} 
	\BRitem[Tipo:] Restricción (validación).
	\BRitem[Descripción:]Para solicitar un préstamo o prestar un libro a un usuario de otra biblioteca, la biblioteca externa debe tener convenio con la nuestra y estar dada de alta en el sistema. 
\end{BussinesRule}

\begin{BussinesRule}{RN3.16}{Préstamo a Lector Externo} 
	\BRitem[Tipo:] Restricción (validación).
	\BRitem[Descripción:]El lector externo solo se le permite pedir “libros” en calidad de préstamo.
\end{BussinesRule}

\begin{BussinesRule}{RN3.17}{Limite de Préstamos Interbibliotecarios a Lector } 
	\BRitem[Tipo:] Restricción (validación).
	\BRitem[Descripción:]El lector se le permite solo 2 préstamos.
\end{BussinesRule}

\begin{BussinesRule}{RN3.18}{Limite de Préstamo Interbibliotecario por documento} 
	\BRitem[Tipo:] Restricción (validación).
	\BRitem[Descripción:]Solo se permite el préstamo de un libro por cada formato interbibliotecario.
\end{BussinesRule}

\begin{BussinesRule}{RN3.19}{Registro Devolución de Material } 
	\BRitem[Tipo:] Restricción (validación).
	\BRitem[Descripción:]Al registrar la devolución del material, el estado de este pasara a “disponible”
\end{BussinesRule}

\begin{BussinesRule}{RN3.20}{Estado Fisico del Material} 
	\BRitem[Tipo:] Restricción (validación).
	\BRitem[Descripción:]Si el material es regresado en mal estado y no existían observaciones sobre esto previamente, se asumirá que el usuario las causó y se generará una penalización monetaria. A continuación, se muestra una tabla describiendo el estado en que el libro es regresado y el costo que genera.
\end{BussinesRule}

\begin{BussinesRule}{RN3.21}{Pérdida de Material} 
	\BRitem[Tipo:] Restricción (validación).
	\BRitem[Descripción:]Se le generara al usuario una multa por el costo total del material en calidad de “Perdido”
\end{BussinesRule}

\begin{BussinesRule}{RN3.22}{Devolución a destiempo de libros y M. Audiovisual} 
	\BRitem[Tipo:] Restricción (validación).
	\BRitem[Descripción:]El usuario pagará 6 pesos por cada día de retardo en la entrega del material.
\end{BussinesRule}

\begin{BussinesRule}{RN3.23}{Devolución a destiempo de Material de TTs} 
	\BRitem[Tipo:] Restricción (validación).
	\BRitem[Descripción:]El usuario pagará 6 pesos por cada hora de retardo en la entrega de un Material de TT. 
\end{BussinesRule}

\begin{BussinesRule}{RN3.24}{ID Multas} 
	\BRitem[Tipo:] Restricción (validación).
	\BRitem[Descripción:]La multa contara con un ID propio e irrepetible, fecha de expedición, los datos del usuario acreedor y el concepto de la multa.


\end{BussinesRule}

\begin{BussinesRule}{RN3.25}{Multa Pagada} 
	\BRitem[Tipo:] Restricción (validación).
	\BRitem[Descripción:]Al registrarse el pago de la multa el estado de esta será cambiado a “Pagada”.
\end{BussinesRule}

\begin{BussinesRule}{RN3.26}{Cancelar Multa} 
	\BRitem[Tipo:] Restricción (validación).
	\BRitem[Descripción:]Solo se puede cancelar una multa cuando el concepto de esta es por Perdida de Libro y no ha sido pagada.
\end{BussinesRule}

\begin{BussinesRule}{RN3.27}{Consulta de Préstamo Internos e Interbibliotecarios} 
	\BRitem[Tipo:] Restricción (validación).
	\BRitem[Descripción:]El jefe de la biblioteca es el único que puede consultar los préstamos que se han hecho y los prestamos interbibliotecarios que se han pedido.
\end{BussinesRule}

\begin{BussinesRule}{RN3.28}{Datos de la Consulta} 
	\BRitem[Tipo:] Restricción (validación).
	\BRitem[Descripción:]Los datos introducidos para la consulta de préstamos internos e interbibliotecarios deben de existir en la Base de Datos.
\end{BussinesRule}

\begin{BussinesRule}{RN3.30}{Limite de prestamos interbibliotecarios.} 
	\BRitem[Tipo:] Restricción (validación).
	\BRitem[Descripción:]El Usuario interno podrá tener máximo dos prestamos interbibliotecarios
\end{BussinesRule}

\begin{BussinesRule}{RN3.31}{Semestre del Usuario.} 
	\BRitem[Tipo:] Restricción (validación).
	\BRitem[Descripción:]Para solicitar un préstamo interbibliotecario el usuario debe estar en 5 semestre o superior.
\end{BussinesRule}
%---------------------------------------------------------

\begin{BussinesRule}{RN4.1}{Campos no nulos.} 
	%\BRitem[Autor:] Miguel Ángel Castañeda Sánchez.
	\BRitem[Descripción:] Ningún dato en el formulario del lector puede ser nulo.
	\BRitem[Tipo:] Restricción (validación).
	\BRitem[Nivel:] Obligatorio.
\end{BussinesRule}

%---------------------------------------------------------

\begin{BussinesRule}{RN4.2}{Formato del número de boleta del IPN.}
	%\BRitem[Autor:] Miguel Ángel Castañeda Sánchez
	\BRitem[Descripción:] La boleta está compuesta por:
		\begin{itemize} 
			\item Año de ingreso del estudiante(4 dígitos).
			\item Número de matrícula de la escuela(2 dígitos).
			\item Número de estudiante(4 dígitos).
		\end{itemize}
	Ejemplo:\\
		Año de ingreso: 2014\\
		Escom: 63\\
		No. Estudiante: 147\\
		Su boleta seŕa: 2014630147\\
	\BRitem[Tipo:] Restricción (validación).
	\BRitem[Nivel:] Obligatorio.
\end{BussinesRule}

%---------------------------------------------------------

\begin{BussinesRule}{RN4.3}{Formato del nombre.}
	%\BRitem[Autor:] Miguel Ángel Castañeda Sánchez.
	\BRitem[Descripción:] El nombre esta compuesto por:
		\begin{itemize} 
			\item Nombre
			\item Primer apellido
			\item Segundo apellido 
		\end{itemize}
		Todo el nombre debe estar compuestos por letras.\\\\
Ejemplo \\
	Nombre: Luis Ángel, Apellido Paterno: Martínez, Apellido Materno: Gómez.
	\BRitem[Tipo:] Restricción(validación)
	\BRitem[Nivel:] Obligatorio.
\end{BussinesRule}

%---------------------------------------------------------

\begin{BussinesRule}{RN4.4}{Formato del CURP.}
	%\BRitem[Autor:] Miguel Ángel Castañeda Sánchez.
	\BRitem[Descripción:] El formato del CURP esta compuesto por::
		\begin{itemize}
			\item Primera letra y la primera vocal del primer apellido,
			\item Primera letra del segundo apellido,
			\item Primera letra del nombre,
			\item Fecha de nacimiento sin espacios en orden de año, mes y dia; ejemplo 940608 (08 de Junio de 1994),
			\item letra del sexo (H o M);
			\item Dos letras correspondientes a la entidad de nacimiento;
			\item Primera consonante interna (no inicial) del primer apellido;
			\item Primera consonante interna (no inicial) del segundo apellido;
			\item Primera consonante interna (no inicial) del nombre,
			\item Dígito del 0-9 para fechas de nacimiento hasta el año 1999 y A-Z para fechas de nacimiento a partir del 2000,
			\item Dígito, para evitar duplicaciones.			
		\end{itemize}

Ejemplo:\\
	Nombre: Luis Ángel Sánchez Fernańdez, Sexo: Masculino, Fecha de nacimiento: 05 de Julio de 1994 y Estado: Colima.\\
	Su CURP será: SAFL940705HCMNRS09.

	\BRitem[Tipo:] Restricción(validación)
	\BRitem[Nivel:] Obligatorio.
\end{BussinesRule}

%---------------------------------------------------------

\begin{BussinesRule}{RN4.5}{Formato de la fecha.}
	%\BRitem[Autor:] Miguel Ángel Castañeda Sánchez.
	\BRitem[Descripción:] La fecha tiene el siguiente formato :
		\begin{itemize} 
			\item Dia(DD)/ 
			\item Mes(MM)/
			\item Año (AAAA)
		\end{itemize}
Ejemplo \\
	16/05/2017
	\BRitem[Tipo:] Restricción(validación)
	\BRitem[Nivel:] Obligatorio.
\end{BussinesRule}

%---------------------------------------------------------

\begin{BussinesRule}{RN4.6}{Formato de la dirección.} 
	%\BRitem[Autor:] Miguel Ángel Castañeda Sánchez.
	\BRitem[Descripción:]El formato de la dirección debe ser:
		\begin{itemize}
			\item Tipo y nombre de la vialidad, 
			\item Número del domicilio, 
			\item Colonia, 
			\item Código postal, 
			\item Municipio, 
			\item Entidad federativa.
		\end{itemize}
	Ejemplo: Av. Rosales, No.5217, Col. Panamericana, C.P. 07770, Naucalpan de Juárez, Puebla.	
	\BRitem[Tipo:] Restricción (validación).
	\BRitem[Nivel:] Obligatorio.
\end{BussinesRule}

%---------------------------------------------------------

\begin{BussinesRule}{RN4.7}{Formato de teléfono.} 
	%\BRitem[Autor:] Miguel Ángel Castañeda Sánchez.
	\BRitem[Descripción:] El teléfono debe de estar formado solamente con números.
	\BRitem[Tipo:] Restricción (validación).
	\BRitem[Nivel:] Obligatorio.
\end{BussinesRule}

%---------------------------------------------------------

\begin{BussinesRule}{RN4.8}{Formato del semestre.} 
	%\BRitem[Autor:] Miguel Ángel Castañeda Sánchez.
	\BRitem[Descripción:] El formato del semestre será elegido desde un menú en el cual se podra elegir la opción que va desde:
		\begin{itemize}
			\item 1er semestre
			\item 2do semestre
			\item .
			\item ..
			\item ...
			\item 10mo semestre en adelante
		\end{itemize}
	\BRitem[Tipo:] Restricción (validación).
	\BRitem[Nivel:] Obligatorio.
\end{BussinesRule}

%---------------------------------------------------------

\begin{BussinesRule}{RN4.9}{Formato del email.} 
	%\BRitem[Autor:] Miguel Ángel Castañeda Sánchez.
	\BRitem[Descripción:] El formato del email puede estar formado por letras, numeros.\\
Ejemplo:\\
	luis-calles@hotmail.com\\
	xxxxx@xxxx.com\\

	\BRitem[Tipo:] Restricción (validación).
	\BRitem[Nivel:] Obligatorio.
\end{BussinesRule}

%---------------------------------------------------------

\begin{BussinesRule}{RN4.10}{Formato de la contraseña.} 
	%\BRitem[Autor:] Miguel Ángel Castañeda Sánchez.
	\BRitem[Descripción:] La contraseña debe de tener un tamaño mínimo de 8 caracteres y un máximo de 16 caracteres, la cual está compuesta por:
		\begin{itemize}	
			\item Letras mayúsculas.
			\item Letras minúsculas.
			\item Dígitos.
			\item Caracteres no alfanúmericos, es decir, caracteres especiales.
		\end{itemize}
	\BRitem[Tipo:] Restricción (validación).
	\BRitem[Nivel:] Obligatorio.
\end{BussinesRule}

%---------------------------------------------------------

\begin{BussinesRule}{RN4.11}{Formato de la credencial.} 
	%\BRitem[Autor:] Miguel Ángel Castañeda Sánchez.
	\BRitem[Descripción:] El formato de la credencial debe de contener los siguientes datos:
		\begin{itemize}	
			\item ID del lector.
			\item Nombre completo del lector.
		\end{itemize}
	\BRitem[Tipo:] Restricción (validación).
	\BRitem[Nivel:] Obligatorio.
\end{BussinesRule}

%---------------------------------------------------------

\begin{BussinesRule}{RN4.12}{Formato del número de empleado del IPN.}
	%\BRitem[Autor:] Miguel Ángel Castañeda Sánchez
	\BRitem[Descripción:] El número del empleado esta compuesto por:
		\begin{itemize} 
			\item Año de ingreso del docente(4 dígitos).
			\item Dos dígitos para saber de donde es egresado(01 egresado del IPN, 10 egresado de otra universidad).
			\item Número de trabajador(4 dígitos).
		\end{itemize}
	Ejemplo:\\
		Año de ingreso: 2014\\
		Egersado de ESCOM: 01\\
		No. trabajador: 3147\\
		Su número de trabajador: 201403147\\
	\BRitem[Tipo:] Restricción (validación).
	\BRitem[Nivel:] Obligatorio.
\end{BussinesRule}

%---------------------------------------------------------

\begin{BussinesRule}{RN4.13}{Formato del departamento.} 
	%\BRitem[Autor:] Miguel Ángel Castañeda Sánchez.
	\BRitem[Descripción:] El formato del departamento será elegido desde un menú en el cual se podra elegir la opción que va desde:
		\begin{itemize}
			\item Formación básica
			\item Ciencias e ingeniería de la computación
			\item Ingeniería en sistemas computacionales
			\item Formación integral e institucional
		\end{itemize}
	\BRitem[Tipo:] Restricción (validación).
	\BRitem[Nivel:] Obligatorio.
\end{BussinesRule}

%---------------------------------------------------------

\begin{BussinesRule}{RN4.14}{Lector no registrado en el sistema} 
	%\BRitem[Autor:] Miguel Ángel Castañeda Sánchez.
	\BRitem[Descripción:] Verificar que el lector no este dado de alta el sistema. Prop+osito evitar duplicidad de lectores en el sistema.
	\BRitem[Tipo:] Restricción (validación).
	\BRitem[Nivel:] Obligatorio.
\end{BussinesRule}

%---------------------------------------------------------

\begin{BussinesRule}{RN4.15}{Lector no registrado en el sistema} 
	%\BRitem[Autor:] Miguel Ángel Castañeda Sánchez.
	\BRitem[Descripción:] Verificar que el lector no este dado de alta el sistema. Propósito evitar duplicidad de lectores en el sistema.
	\BRitem[Tipo:] Restricción (validación).
	\BRitem[Nivel:] Obligatorio.
\end{BussinesRule}

%---------------------------------------------------------

\begin{BussinesRule}{RN4.16}{Alumno vigente en el instituto} 
	%\BRitem[Autor:] Miguel Ángel Castañeda Sánchez.
	\BRitem[Descripción:] Verificar que el alumno este inscrito en el instituto en el semestre actual.
	\BRitem[Tipo:] Restricción (validación).
	\BRitem[Nivel:] Obligatorio.
\end{BussinesRule}

%---------------------------------------------------------

\begin{BussinesRule}{RN4.17}{Registro de alta del lector llenado correctamente } 
	%\BRitem[Autor:] Miguel Ángel Castañeda Sánchez.
	\BRitem[Descripción:] Verificar que el registro del lector sea llenado correctamente. Que la información ingresada en el formulario corresponda a los formatos de cada campo.
	\BRitem[Tipo:] Restricción (validación).
	\BRitem[Nivel:] Obligatorio.
\end{BussinesRule}

%---------------------------------------------------------

\begin{BussinesRule}{RN4.18}{Registro de alta del lector llenado correctamente } 
	%\BRitem[Autor:] Miguel Ángel Castañeda Sánchez.
	\BRitem[Descripción:] Verificar que el registro del lector sea llenado correctamente. Que la información ingresada en el formulario corresponda a los formatos de cada campo.
	\BRitem[Tipo:] Restricción (validación).
	\BRitem[Nivel:] Obligatorio.
\end{BussinesRule}

%---------------------------------------------------------

\begin{BussinesRule}{RN4.19}{Comprobar identidad del lector } 
	%\BRitem[Autor:] Miguel Ángel Castañeda Sánchez.
	\BRitem[Descripción:] Verificar que la identificacion del lector corresponda con los datos del perfil del lector.
	\BRitem[Tipo:] Restricción (validación).
	\BRitem[Nivel:] Obligatorio.
\end{BussinesRule}

%---------------------------------------------------------

\begin{BussinesRule}{RN4.20}{El lector tiene multas} 
	%\BRitem[Autor:] Miguel Ángel Castañeda Sánchez.
	\BRitem[Descripción:] El lector no puede actualizar sus datos hasta que no pague las multas pendientes que tenga.
	\BRitem[Tipo:] Restricción (validación).
	\BRitem[Nivel:] Obligatorio.
\end{BussinesRule}

%---------------------------------------------------------

\begin{BussinesRule}{RN4.21}{El lector tiene adeudos de material} 
	%\BRitem[Autor:] Miguel Ángel Castañeda Sánchez.
	\BRitem[Descripción:] El lector no puede actualizar sus datos hasta que no entregue el material adeudado que tenga. Con adeudado se refiere a material entregado que no ha sido entregado en su fecha de entrega.
	\BRitem[Tipo:] Restricción (validación).
	\BRitem[Nivel:] Obligatorio.
\end{BussinesRule}

%---------------------------------------------------------

\begin{BussinesRule}{RN4.22}{Lector dado de alta} 
	%\BRitem[Autor:] Miguel Ángel Castañeda Sánchez.
	\BRitem[Descripción:] El lector esta dado de alta en el sistema.
	\BRitem[Tipo:] Restricción (validación).
	\BRitem[Nivel:] Obligatorio.
\end{BussinesRule}

%---------------------------------------------------------

\begin{BussinesRule}{RN4.23}{Lector sin credencial.}
	\BRitem[Tipo:] Restricción (validación).
	%\BRitem[Autor:] Miguel Ángel Castañeda Sánchez.
	\BRitem[Descripción:] El lector que extravio su credencial, puede renovar su crendecial.
\end{BussinesRule}

%---------------------------------------------------------

