% Copie este bloque por cada caso de uso:
%-------------------------------------- COMIENZA descripción del caso de uso.

\begin{UseCase}{CU2.4C}{Consultar Equipo Audiovisual}{
	Se podrá consultar la información de un material de tipo Equipo audiovisual previamente registrado en la biblioteca.
}
		\UCitem{Versión}{0.1}
		\UCitem{Actor}{Procesos Técnicos}
		\UCitem{Propósito}{Mostrar la información de un Equipo Audiovisual que ha sido registrado en la biblioteca.}
		\UCitem{Entradas}{
			\begin{itemize}
		   		\item ID Audiovisual
		   		\item Título
		  	\end{itemize}					
		}


		\UCitem{Salidas}{
			\begin{itemize}
			\item ID	
			\item Titulo	
			\item Autor(es)	
			\item Fecha de Publicacion	
			\item Duracion	
			\item Existencias	
			\item Precio	
			\item Tipo
			\item Mensajes para el usuario
				\begin{itemize}
		   			\item \MSGref{MSJ2.1b1}{Campos incompletos}
		   			\item \MSGref{MSJ2.1b3}{Error en conexión}
		   			\item \MSGref{MSJ2.1c1}{El material no está registrado} 
		   		\end{itemize}
		  	\end{itemize} 
		}


		\UCitem{Precondiciones}{ 
			\begin{itemize} 
				\item Viene como extends del \UCref{CU2.4}
				\item El sistema debe de estar conectado a la base de datos
				\item El Material audivisual a consultar debe haber sido registrado anteriormente
			\end{itemize}
		}
		\UCitem{Postcondiciones}{
			\begin{itemize}
				\item Ninguna
			\end{itemize}			
		}
		\UCitem{Autor}{
			\begin{itemize}
				\item Cortés Frias Diego Antonio
			\end{itemize}					
		}
		\UCitem{Revisó}{
			\begin{itemize}
				\item Ro
			\end{itemize}					
		}
		\UCitem{Estatus}{
			\begin{itemize}
				\item En revisión
			\end{itemize}					
		}
\end{UseCase}

%----------------- COMIENZA la descripción de Trayectoria Principal

\begin{UCtrayectoria}{Principal}
		\UCpaso[\UCactor] Proporciona el Número del equipo audiovisual en el campo mostrado en pantalla \IUref{IU2.C1C}{ConsultarEquipoAudiovisual}
		\UCpaso[\UCactor] Presiona el botón \IUbutton{consultar}
		\UCpaso[\UCsist] Verifica que se haya llenado el campo requerido \Trayref{A}
		\UCpaso[\UCsist] Busca el dato ingresado en la base de datos \Trayref{B} y verifica que exista \Trayref{C}
		\UCpaso[\UCsist] Muestra una tabla con la información del material que está almacenada en la base de datos.
\end{UCtrayectoria}


%--------------- COMIENZAN las Trayectorias Alternativas
%---------- TRAYECTORIA A

\begin{UCtrayectoriaA}{A}{Se viola la regla de negocio \BRref{RN2.1b1}{Campos incompletos}}	
			\UCpaso[\UCsist] Informa al actor mostrando el mensaje de error \MSGref{MSJ2.1b1}{Campos incompletos} en los campos que sean obligatorios de llenar
			\UCpaso[\UCactor] Proporciona los datos faltantes en los campos marcados 
			\UCpaso[\UCactor] Presiona el botón \IUbutton{consultar}
			\UCpaso[\UCsist] Continúa trayectoria con el siguiente paso de la trayectoria principal \Trayref{Principal}
\end{UCtrayectoriaA}

%---------- TRAYECTORIA B
\begin{UCtrayectoriaA}{C}{Falló la conexion en la base de datos}
			\UCpaso[\UCsist] El sistema mostrará el mensaje \MSGref{MSJ2.1b3}{Error en conexión}
			\UCpaso[\UCsist] Regresa a un paso atrás de la trayectoria principal \Trayref{Principal}
\end{UCtrayectoriaA}

%---------- TRAYECTORIA C
\begin{UCtrayectoriaA}{D}{Se viola la \BRref{RN2.C11}{El material no está registrado}}
			\UCpaso[\UCsist] El sistema mostrará el mensaje \MSGref{MSJ2.1c1}{El material no está registrado}
			\UCpaso[\UCsist] Regresa a un paso atrás de la trayectoria principal \Trayref{Principal}
\end{UCtrayectoriaA}





