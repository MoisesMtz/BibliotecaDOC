% Copie este bloque por cada caso de uso:
%-------------------------------------- COMIENZA descripción del caso de uso.

\begin{UseCase}{CU2.6A}{Modificar libro}{
	Este caso de uso se podrá modificar un  libro para tener la información necesaria en el catálogo de libros identificados por el ISBN, y a su vez, haciéndolos únicos asignando un ID para cada uno, el cual, será impreso
}
		\UCitem{Versión}{0.1}
		\UCitem{Actor}{Procesos Técnicos}
		\UCitem{Propósito}{La información de un libro se modificara en el sistema}
		\UCitem{Entradas}{
			\begin{itemize}
		   		\item ISBN del libro
		   		\item Título
		   		\item Autor
		   		\item Editorial
		   		\item Fecha de publicación
		   		\item Número de páginas
		   		\item Edición
		   		\item Precio
		   		\item Cantidad a registrar
		  	\end{itemize}					
		}
		\UCitem{Salidas}{
			\begin{itemize}
				\item Mensajes para el usuario
				\begin{itemize}
		   			\item \MSGref{MSJ2.1b1}{Campos incompletos}
		   			\item \MSGref{MSJ2.1b2}{Formato inválido}
		   			\item \MSGref{MSJ2.1b3}{Error en conexión}
		   			\item \MSGref{MSJ2.1b4}{Modificación existosa} 
		   		\end{itemize}
		   		\item Formato de los Identificadores generados para imprimir
		  	\end{itemize} 
		}
		\UCitem{Precondiciones}{ 
			\begin{itemize} 
				\item Viene como extends del \UCref{CU2.6b}
				\item El sistema debe de estar conectado a la base de datos
			\end{itemize}
		}
		\UCitem{Postcondiciones}{
			\begin{itemize}
				\item Se actualiza la base de datos con el libro modificado
			\end{itemize}			
		}
		\UCitem{Autor}{
			\begin{itemize}
				\item López Rojas Guillermo
			\end{itemize}					
		}
		\UCitem{Revisó}{
			\begin{itemize}
				\item 
			\end{itemize}					
		}
		\UCitem{Estatus}{
			\begin{itemize}
				\item 
			\end{itemize}					
		}
\end{UseCase}

%----------------- COMIENZA la descripción de Trayectoria Principal

\begin{UCtrayectoria}{Principal}
		\UCpaso[\UCactor] Proporciona el ISBN del libro en el campo mostrado en pantalla \IUref{IU2.1ba}{ConsultarLibro}
		\UCpaso[\UCactor] Presiona el botón \IUbutton{Continuar} \Trayref{Limpiar}
		\UCpaso[\UCsist] Verifica que se cumpla la regla de negocio \BRref{RN2.1b1}{Campos incompletos} \Trayref{A}
		\UCpaso[\UCsist] Verifica que se cumpla la regla de negocio \BRref{RN2.1b2}{Formato inválido} \Trayref{B}
		\UCpaso[\UCsist] Busca el dato ingresado en la base de datos \Trayref{C} 
		\UCpaso[\UCactor] Presiona el botón \IUbutton{Modificar Catalogo} 
		\UCpaso[\UCsist] Despliega la pantalla \IUref{IU2.1ba1}{Modificar Libro}
		\UCpaso[\UCactor] Proporciona los datos requeridos en los campos que se muestran en la pantalla. 
		\UCpaso[\UCactor] Presiona el botón \IUbutton{Modificar}
		\UCpaso[\UCsist] Verifica que se cumpla la regla de negocio \BRref{RN2.1b1}{Campos incompletos} \Trayref{A}
		\UCpaso[\UCsist] Verifica que se cumpla la regla de negocio \BRref{RN2.1b2}{Formato inválido} \Trayref{B}
		\UCpaso[\UCsist] Modifica los datos del libro en la base de datos
		\UCpaso[\UCsist] Muestra el mensaje \MSGref{MSJ2.1b4}{Modificación existosa}
		\UCpaso[\UCactor] Confirma operación haciendo click en el botón \IUbutton{Confirmar} en la ventana emergente que muestra.
\end{UCtrayectoria}


%--------------- COMIENZAN las Trayectorias Alternativas
%---------- TRAYECTORIA A

\begin{UCtrayectoriaA}{A}{Se viola la regla de negocio \BRref{RN2.1b1}{Campos incompletos}}	
			\UCpaso[\UCsist] Informa al actor mostrando el mensaje de error \MSGref{MSJ2.1b1}{Campos incompletos} en los campos que sean obligatorios de llenar
			\UCpaso[\UCactor] Proporciona los datos faltantes en los campos marcados 
			\UCpaso[\UCactor] Presiona el botón \IUbutton{Registrar} \Trayref{Limpiar}
			\UCpaso[\UCactor] Verifica que se cumpla la regla de negocio \BRref{RN2.1b1}{Campos incompletos} \Trayref{A}
			\UCpaso[\UCsist] Continúa trayectoria con el siguiente paso de la trayectoria que invocó esta trayectoria alternativa
\end{UCtrayectoriaA}

%---------- TRAYECTORIA B

\begin{UCtrayectoriaA}{B}{Se viola la regla de negocio \BRref{RN2.1b2}{Formato inválido}}	
			\UCpaso[\UCsist] Informa al actor mostrando el mensaje de error \MSGref{MSJ2.1b2}{Formato inválido} en los campos donde esté incorrecto el formato
			\UCpaso[\UCactor] Proporciona los datos correctos en los campos marcados 
			\UCpaso[\UCactor] Presiona el botón \IUbutton{Registrar} \Trayref{Limpiar}
			\UCpaso[\UCsist] Verifica que se cumpla la regla de negocio \BRref{RN2.1b1}{Campos incompletos} \Trayref{A}
			\UCpaso[\UCsist] Verifica que se cumpla la regla de negocio \BRref{RN2.1b2}{Formato inválido} \Trayref{B}
			\UCpaso[\UCsist] Continúa trayectoria con el siguiente paso de la trayectoria que invocó esta trayectoria alternativa
\end{UCtrayectoriaA}

%---------- TRAYECTORIA C

\begin{UCtrayectoriaA}{C}{Falló la conexion en la base de datos}
			\UCpaso[\UCsist] El sistema mostrará el mensaje \MSGref{MSJ2.1b3}{Error en conexión}
			\UCpaso[\UCsist] Regresa al paso 1 de la trayectoria principal \Trayref{Principal} 
\end{UCtrayectoriaA}


%---------- TRAYECTORIA Limpiar

\begin{UCtrayectoriaA}{Limpiar}{El actor presiona el botón \IUbutton{Limpiar}}
			\UCpaso[\UCsist] Eliminar todos los datos ingresados en los campos mostrados en la pantalla actual y mostrarlos vacíos
			\UCpaso[\UCsist] Continúa trayectoria con un paso anterior de la trayectoria que invocó esta trayectoria alternativa
\end{UCtrayectoriaA}